\documentclass[xcolor=dvipsnames]{beamer} 
 
\usetheme{Goettingen}
% AnnArbor, Antibes, Bergen, Berkeley, Berlin, Boadilla, CambridgeUS, Copenhagen, 
% Darmstadt, default, Dresden, Frankfurt, Goettingen, Hannover, Ilmenau, JuanLesPins, 
% Luebeck, Madrid, Malmoe, Marburg, Montpellier, PaloAlto, Pittsburgh, Rochester, 
% Singapore, Szeged, Warsaw
 
\usefonttheme{serif}
% serif, euler, newcent, avant, helvet, palatino, bookman, mathtime, pifont, chancery, charter, mathptm, mathptmx, utopia
 
\usecolortheme[named=RoyalBlue]{structure}
% or \usercolortheme{lily}
% albatross, beaver, beetle, crane, default, dolphin, dove, fly, lily, orchid, rose, seagull, seahorse, whale, wolverine
 
% see http://language.dyndns.org/research/latex/docs/beamerColor.pdf
\setbeamertemplate{items}[ball] 
\setbeamertemplate{blocks}[rounded][shadow=true] 
 
\usepackage{amsthm,amsmath,amsfonts} % AMSLaTeX packages
\usepackage{graphicx} % we want to use images
 
% PDF settings
\hypersetup{%
    pdftitle={R for Exploratory Data Analysis},%
    pdfauthor={Jim Crozier},%
    pdfsubject={},%
    pdfkeywords={}%
}
 
% \mode
% \mode
\setbeamertemplate{footline} {%
  \leavevmode%
  \hbox{\begin{beamercolorbox}[wd=.5\paperwidth,ht=2.5ex,dp=1.125ex,leftskip=.3cm]{author in head/foot}%
    \usebeamerfont{author in head/foot}\insertshortauthor \end{beamercolorbox}%
  \begin{beamercolorbox}[wd=.5\paperwidth,ht=2.5ex,dp=1.125ex,leftskip=.3cm,rightskip=.3cm plus1fil]{title in head/foot}%
    \usebeamerfont{title in head/foot}\insertshortinstitute \hfill \insertframenumber/\inserttotalframenumber
  \end{beamercolorbox}}%
  \vskip0pt%
}
 
% \logo{\includegraphics[width=0.3in]{latexImg/GCLogoShort.eps}}
 
\usepackage{tree-dvips,graphicx,color,qtree,apacite,algorithm,algorithmic,multirow,tipa,vowel}
 
\newcommand{\subscript}[1]{{\textnormal{\scriptsize{#1}}}}  % for writing subscript text
 
% from http://blog.miliauskas.lt/2008/09/python-syntax-highlighting-in-latex.html
% http://www.ctan.org/tex-archive/macros/latex/contrib/listings/listings.pdf
 
\usepackage{textcomp}
\usepackage{setspace}
 
% \usepackage{cite}
\usepackage{natbib}
\renewcommand{\bibfont}{\tiny}
 

 
%%%%%%%%%%%%%%%%%%%%
% the paper title
%%%%%%%%%%%%%%%%%%%%
\title{R for Exploratory Data Analysis}
% \subtitle{}
\author{Jim Crozier}
\institute[Stratusfy LLC jim@stratusfy.com]{Stratusfy: Data Science and Cloud Technology.}
\date{\today}
 
%%%%%%%%%%%%%%%%%%%%
% body
%%%%%%%%%%%%%%%%%%%%
\usepackage{Sweave}
\begin{document}
\Sconcordance{concordance:presentation.tex:presentation.Rnw:%
1 72 1 1 0 112 1}

 
\begin{frame}
% \begin{frame}[allowframebreaks,allowdisplaybreaks]
% \begin{frame}[shrink=5]
    \titlepage
\end{frame}
 
\section{Introduction}
\begin{frame}[t,allowframebreaks]{Overview}
\begin{block}{Quick Notes}
Code and documentation available at github/jimcrozier/r-meetup-rattle
\end{block}

\begin{block}{Topics}
Three topics for data exploration:
\end{block}
 
\begin{itemize}
    \item rattle package
    \item knitr package
    \item shiny application
\end{itemize}

\begin{block}{Bonus:}
Bonus: R + javascript for geomapping
\end{block}
 
\end{frame}

\section{Topics: rattle}
\begin{frame}[t,allowframebreaks]{rattle}
\begin{block}{rattle Overview }
rattle is a free gui available from Togaware for R
\end{block}

\begin{itemize}
\item I am not a rattle expert
\item Loads of example available online
\item Covering the very basics here
\end{itemize}
\begin{block}{Getting started }
Install and load gui
\end{block}

install.packages(rattle, dependencies=T)

\begin{itemize}
\item This takes some time due to the large number of dependencies
\end{itemize}
rattle()

\begin{block}{Make it work}
Live demo
\end{block}
data(iris)

\end{frame}


\section{Topics: knitr}
\begin{frame}[t,allowframebreaks]{knitr}
\begin{block}{Overview }
knitr is a single best tool that you can learn
\end{block}
\begin{block}{Overview }
copy and paste rattle log into block
\end{block}

\begin{itemize}
\item Convert crs and crv to list()
\item Still working out kinks with plots
\end{itemize}
\begin{block}{Overview }
Tip: use cat() to write out .tex and looping to create integrated EDA
\end{block}
"Intelligence is the faculty of making artificial objects, especially tools to make tools." - Henri Bergson
\end{frame}


\section{Topics: shiny}
\begin{frame}[t,allowframebreaks]{shiny}
\begin{block}{Make it work}
Interactivity is where data science is headed. shiny is easy way to interact your models and your views. 
\end{block}

\begin{block}{Make it work}
Live demo 
\end{block}

\end{frame}

\section{Bonus: javascript with R}
\begin{frame}[t,allowframebreaks]{Javascript with R}

\begin{block}{Make it work}
Use R, JSON and javascript (here the leaf package) to create interactive elements
\end{block}

\begin{block}{Make it work}
Live demo Rickshaw
\end{block}

\begin{block}{Make it work}
Live demo Leaf
\end{block}

\end{frame}


 
\end{document}
